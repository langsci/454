\chapter{Preface}

\label{ch:preface}

Much of today's speech and writing is channeled digitally, and computers are increasingly able to disseminate, organize, and produce language along with the information contained therein.
  Such capabilities make people's lives easier: Now everyone on the internet can correspond instantly with billions of people around the globe, and  access information far surpassing even the best research
libraries of the past.   On the other hand, these  tools may trigger a socio-economic transformation with losers as well as winners, and can harm people if used uncritically.

%\wdm{Anything we should add to reflect the developments in the last decade (e.g., the massive social media expansion, hate speech and fake news challenges, ethical aspects of trained models,  widespread use of digital assistants and dialogue systems in cars, \ldots}

This book takes you on a tour of different real-world tasks and
applications where computers deal with language.  
During this tour,
you'll encounter essential concepts relating to language, representation,
and processing, so that by the end of the book you will have a good
grasp of key concepts in the field of language technology (LT),  computational linguistics (CL), and natural language processing (NLP) -- essentially names for the same thing, as viewed from the respective perspectives of industry, linguistics, and computer science.  The
only background you need to read this book is some curiosity about
language and some everyday experience with computers.

%These applications were chosen because: a) they are representative of techniques used throughout the field; b) they represent a significant body of work in and of themselves; c) they connect directly to linguistic modeling; and d) they are the ones the authors know best.  We hope that you will be able to use these examples as an introduction to general concepts which you can apply to learning about other applications and areas of inquiry.

We explore tools that support writing; foreign language learning; the distillation of information from text for research and business purposes; automatic detection of spam, anti-social content, or emotional sentiment; web search;  machine translation; and conversation.  We assume that most of you will be familiar with 
these applications and may wonder how they work or why they don't.  What
you may not realize is how similar the underlying processing is.  For
example, there is a lot in common between how your email system filters spam and how a dialog system identifies what you are asking it to do.  By seeing these concepts recur --
in this case, a machine learning technique called classification -- we hope this will
reinforce the importance of applying general techniques for new
applications. %\wdm{Is this link still visible in the book with its restructured n-gram discussion?}


This book aims to welcome humanities-confident students to learn technical tools, and to invite computing-confident students to appreciate the social richness of language data.    From the technical side, we explore the computational underpinnings of language technology; from the humanistic side, we emphasize the nuances of the linguistic data as well as social, economic, and ethical effects of such technology.  Whether you feel more confident in humanistic or technical ways of thinking, we hope that this book empowers you to combine both approaches to fully understand the value, limitations, and consequences of language technology.


\section*{How to use the book}

There are a number of features in this textbook which allow you to
structure what you learn, explore more about the topics and to
reinforce what you are learning.  As a start, relevant keywords are typeset in bold and shown in
the margins of each page. You can also look those up in the
\emph{Subject index} at the end of the book.

The \emph{Under the Hood} sections included in many of the chapters
are intended to give you more detail on selected advanced topics.  For
those interested in learning more about language and computers, we
hope that you find these sections enjoyable and enlightening, though
the gist of each chapter can be understood without reading those
sections.

At the end of each chapter, there is a \emph{Checklist} indicating
what you should have learned in reading the chapter.  The
\emph{Exercises} found at the end of each chapter review the
chapter's material and give you opportunities to go beyond it.  Our
hope is that the checklist and exercises help you to get a good grasp
of each of the topics and the concepts involved.  Different exercises will appeal to different students; you are welcome to choose the ones that seem most interesting to you, or challenge yourself to try all of them. 
%We recognize, however, that students from different backgrounds have different skills, and so we have marked each question with an indication of who the question is for.  Most questions are appropriate for all students and thus are marked with ALL; LING questions target those interested in linguistics; CS questions are appropriate for those with a background in computer science; and MATH is appropriate for those wanting to tackle more mathematical challenges.  Of course, you should not feel limited by these markers, as a strong enough desire will generally allow you to tackle most questions.

If you enjoy the topic of a particular chapter, we also encourage you
to make use of the \emph{Further reading} recommendations.  You can
also track the \emph{References} at the end of the book.

For more advanced students, we have had success combining this textbook with a reading list of articles about digital humanities,  computational social science, and corpus linguistics, such as those discussed in \chapref{ch:textasdata}, which students are asked to present in class.  (In one of our classes based on this book, the instructor gives a lecture every Tuesday to set up the topic; Thursdays begin with a clicker quiz to test students' understanding, and the rest of Thursday's class is spent on student presentations of articles). The reading list can be updated every year and customized to the interests of the instructor and students, and the presentations give students a chance to learn from one another while practicing valuable humanistic skills in digesting and communicating research findings.  Advanced students may also be assigned to pursue a final project, engaging with the literature that they have read throughout the semester.


Finally, on the book's companion website \url{https://osf.io/v7uqm/}, we offer an example syllabus and slide decks corresponding to each chapter of this book, which instructors are welcome to use and adapt.

